% !TeX encoding = UTF-8
% !TeX program = xelatex
% !TeX spellcheck = en_US

\documentclass[degree=bachelor,fontset=ubuntu]{thuthesis}
  % 学位 degree:
  %   doctor | master | bachelor | postdoc
  % 学位类型 degree-type:
  %   academic(默认)| professional
  % 语言 language
  %   chinese(默认)| english
  % 字体库 fontset
  %   windows | mac | fandol | ubuntu
  % 建议终版使用 Windows 平台的字体编译


% 论文基本配置,加载宏包等全局配置
% !TeX root = ./thuthesis-example.tex

% 论文基本信息配置

\thusetup{
  %******************************
  % 注意:
  %   1. 配置里面不要出现空行
  %   2. 不需要的配置信息可以删除
  %   3. 建议先阅读文档中所有关于选项的说明
  %******************************
  %
  % 输出格式
  %   选择打印版(print)或用于提交的电子版(electronic),前者会插入空白页以便直接双面打印
  %
  output = print,
  % 格式类型
  %   默认为论文(thesis),也可以设置为开题报告(proposal)
  % thesis-type = proposal,
  %
  % 标题
  %   可使用“\\”命令手动控制换行
  %
  title  = {本科生综合论文训练标题},
  title* = {An Introduction to \LaTeX{} Thesis Template of Tsinghua
            University v\version},
  %
  % 学科门类
  %   1. 学术型
  %      - 中文
  %        需注明所属的学科门类,例如:
  %        哲学、经济学、法学、教育学、文学、历史学、理学、工学、农学、医学、
  %        军事学、管理学、艺术学
  %      - 英文
  %        博士:Doctor of Philosophy
  %        硕士:
  %          哲学、文学、历史学、法学、教育学、艺术学门类,公共管理学科
  %          填写“Master of Arts“,其它填写“Master of Science”
  %   2. 专业型
  %      直接填写专业学位的名称,例如:
  %      教育博士、工程硕士等
  %      Doctor of Education, Master of Engineering
  %   3. 本科生不需要填写
  %
  % degree-category  = {工学硕士},
  % degree-category* = {Master of Science},
  %
  % 培养单位
  %   填写所属院系的全名
  %
  department = {交叉信息研究院},
  %
  % 学科
  %   1. 研究生学术型学位,获得一级学科授权的学科填写一级学科名称,其他填写二级学科名称
  %   2. 本科生填写专业名称,第二学位论文需标注“(第二学位)”
  %
  discipline  = {计算机科学与技术(计算机科学实验班)},
  discipline* = {Computer Science and Technology},
  %
  % 专业领域
  %   1. 设置专业领域的专业学位类别,填写相应专业领域名称
  %   2. 2019 级及之前工程硕士学位论文,在 `engineering-field` 填写相应工程领域名称
  %   3. 其他专业学位类别的学位论文无需此信息
  %
  % professional-field  = {计算机技术},
  % professional-field* = {Computer Technology},
  %
  % 姓名
  %
  author  = {XXX},
  author* = {XXX},
  %
  % 学号
  % 仅当书写开题报告时需要(同时设置 `thesis-type = proposal')
  %
  % student-id = {2000310000},
  %
  % 指导教师
  %   中文姓名和职称之间以英文逗号“,”分开,下同
  %
  supervisor  = {XXX, 教授},
  supervisor* = {Professor Zheng Weimin},
  %
  % 副指导教师
  %
  associate-supervisor  = {XXX, 教授},
  associate-supervisor* = {Professor Chen Wenguang},
  %
  % 联合指导教师
  %
  % co-supervisor  = {某某某, 教授},
  % co-supervisor* = {Professor Mou Moumou},
  %
  % 日期
  %   使用 ISO 格式;默认为当前时间
  %
  % date = {2019-07-07},
  %
  % 是否在中文封面后的空白页生成书脊(默认 false)
  %
  include-spine = false,
  %
  % 密级和年限
  %   秘密, 机密, 绝密
  %
  % secret-level = {秘密},
  % secret-year  = {10},
  %
  % 博士后专有部分
  %
  % clc                = {分类号},
  % udc                = {UDC},
  % id                 = {编号},
  % discipline-level-1 = {计算机科学与技术},  % 流动站(一级学科)名称
  % discipline-level-2 = {系统结构},          % 专业(二级学科)名称
  % start-date         = {2011-07-01},        % 研究工作起始时间
}

% 载入所需的宏包

% 定理类环境宏包
\usepackage{amsthm}
% 也可以使用 ntheorem
% \usepackage[amsmath,thmmarks,hyperref]{ntheorem}

\thusetup{
  %
  % 数学字体
  % math-style = GB,  % GB | ISO | TeX
  math-font  = xits,  % stix | xits | libertinus
}

% 可以使用 nomencl 生成符号和缩略语说明
% \usepackage{nomencl}
% \makenomenclature

% 表格加脚注
\usepackage{threeparttable}

% 表格中支持跨行
\usepackage{multirow}

% 固定宽度的表格。
% \usepackage{tabularx}

% 跨页表格
\usepackage{longtable}

% 算法
\usepackage{algorithm}
\usepackage{algorithmic}

% 量和单位
\usepackage{siunitx}

% 参考文献使用 BibTeX + natbib 宏包
% 顺序编码制
\usepackage[sort]{natbib}
\bibliographystyle{thuthesis-numeric}

% 著者-出版年制
% \usepackage{natbib}
% \bibliographystyle{thuthesis-author-year}

% 生命科学学院要求使用 Cell 参考文献格式(2023 年以前使用 author-date 格式)
% \usepackage{natbib}
% \bibliographystyle{cell}

% 本科生参考文献的著录格式
% \usepackage[sort]{natbib}
% \bibliographystyle{thuthesis-bachelor}

% 参考文献使用 BibLaTeX 宏包
% \usepackage[style=thuthesis-numeric]{biblatex}
% \usepackage[style=thuthesis-author-year]{biblatex}
% \usepackage[style=gb7714-2015]{biblatex}
% \usepackage[style=apa]{biblatex}
% \usepackage[style=mla-new]{biblatex}
% 声明 BibLaTeX 的数据库
% \addbibresource{ref/refs.bib}

% 定义所有的图片文件在 figures 子目录下
\graphicspath{{figures/}}

% 数学命令
\makeatletter
\newcommand\dif{%  % 微分符号
  \mathop{}\!%
  \ifthu@math@style@TeX
    d%
  \else
    \mathrm{d}%
  \fi
}
\makeatother

% hyperref 宏包在最后调用
\usepackage{hyperref}



\begin{document}

% 封面
\maketitle

% 学位论文指导小组、公开评阅人和答辩委员会名单
% 本科生不需要
% \input{data/committee}

% 使用授权的说明
% 本科生开题报告不需要
\copyrightpage
% 将签字扫描后授权文件 scan-copyright.pdf 替换原始页面
% \copyrightpage[file=scan-copyright.pdf]

\frontmatter
\input{data/abstract}

% 目录
\tableofcontents

% 插图和附表清单
% 本科生的插图索引和表格索引需要移至正文之后、参考文献前
% \listoffiguresandtables  % 插图和附表清单(仅限研究生)
\listoffigures           % 插图清单
\listoftables            % 附表清单

% 符号对照表
\input{data/denotation}


% 正文部分
\mainmatter

% !TeX root = ../thuthesis-example.tex

\chapter{引\quad 言}

\section{一级节标题第一条}

此部分是论文主体部分的文字格式示例。

\subsection{二级节标题第一条}

主体部分一般从引言(绪论)开始,以结论结束。

引言(绪论)应包括论文的研究目的、流程和方法等。

论文研究领域的历史回顾,文献回溯,理论分析等内容,应独立成章,用足
够的文字叙述。

主体部分由于涉及的学科、选题、研究方法、结果表达方式等有很大的差异,

不能作统一的规定。但是,必须实事求是、客观真切、准确完备、合乎逻辑、层
次分明、简练可读。

\subsection{二级节标题第二条}

论文中应引用与研究主题密切相关的参考文献。参考文献的写法应遵循国家
标准《信息与文献 参考文献著录规则》(GB/T 7714—2015);符合特定学科的通
用范式,可使用 APA 或《清华大学学报(哲学社会科学版)》格式,且应全文统
一,不能混用。此处是正文中引用参考文献的上标标注示例[1]。

当论文中的字、词或短语,需要进一步加以说明,而又没有具体的文献来源
时,用注释。注释一般在社会科学中用得较多。应控制论文中的注释数量,不宜
过多。由于论文篇幅较长,建议采用文中编号加“脚注”的方式。此处是脚注格
式规范示例\footnote{脚注处序号“①,……,⑩”的字体是“正文”,不是“上标”,序号与脚注内容文字之间空半个汉字
符,脚注的段落格式为:单倍行距,段前空 0 磅,段后空 0 磅,悬挂缩进 1.5 字符;字号为小五号字,汉
字用宋体,外文用 Times New Roman 体。}。

\input{data/chap01}
\input{data/chap02}
\input{data/chap03}
\input{data/chap04}


% 其他部分
\backmatter

% 参考文献
\bibliography{ref/refs}  % 参考文献使用 BibTeX 编译
% \printbibliography       % 参考文献使用 BibLaTeX 编译

% 附录
% 本科生需要将附录放到声明之后,个人简历之前
\appendix
\input{data/appendix-survey}       % 本科生:外文资料的调研阅读报告
\input{data/appendix-translation}  % 本科生:外文资料的书面翻译
\input{data/appendix}

% 致谢
\input{data/acknowledgements}

% 声明
% 本科生开题报告不需要
\statement
% 将签字扫描后的声明文件 scan-statement.pdf 替换原始页面
% \statement[file=scan-statement.pdf]
% 本科生编译生成的声明页默认不加页脚,插入扫描版时再补上;
% 研究生编译生成时有页眉页脚,插入扫描版时不再重复。
% 也可以手动控制是否加页眉页脚
% \statement[page-style=empty]
% \statement[file=scan-statement.pdf, page-style=plain]

% 个人简历、在学期间完成的相关学术成果
% 本科生可以附个人简历,也可以不附个人简历
\input{data/resume}

% 指导教师/指导小组评语
% 本科生不需要
% \input{data/comments}

% 答辩委员会决议书
% 本科生不需要
% \input{data/resolution}

% 本科生的综合论文训练记录表(扫描版)
% \record{file=scan-record.pdf}

\end{document}
