% !TeX root = ../thuthesis-example.tex

\chapter{引\quad 言}

\section{一级节标题第一条}

此部分是论文主体部分的文字格式示例。

\subsection{二级节标题第一条}

主体部分一般从引言(绪论)开始,以结论结束。

引言(绪论)应包括论文的研究目的、流程和方法等。

论文研究领域的历史回顾,文献回溯,理论分析等内容,应独立成章,用足
够的文字叙述。

主体部分由于涉及的学科、选题、研究方法、结果表达方式等有很大的差异,

不能作统一的规定。但是,必须实事求是、客观真切、准确完备、合乎逻辑、层
次分明、简练可读。

\subsection{二级节标题第二条}

论文中应引用与研究主题密切相关的参考文献。参考文献的写法应遵循国家
标准《信息与文献 参考文献著录规则》(GB/T 7714—2015);符合特定学科的通
用范式,可使用 APA 或《清华大学学报(哲学社会科学版)》格式,且应全文统
一,不能混用。此处是正文中引用参考文献的上标标注示例[1]。

当论文中的字、词或短语,需要进一步加以说明,而又没有具体的文献来源
时,用注释。注释一般在社会科学中用得较多。应控制论文中的注释数量,不宜
过多。由于论文篇幅较长,建议采用文中编号加“脚注”的方式。此处是脚注格
式规范示例\footnote{脚注处序号“①,……,⑩”的字体是“正文”,不是“上标”,序号与脚注内容文字之间空半个汉字
符,脚注的段落格式为:单倍行距,段前空 0 磅,段后空 0 磅,悬挂缩进 1.5 字符;字号为小五号字,汉
字用宋体,外文用 Times New Roman 体。}。
